%%%%%%%%%%%%%%%%%%%%%%%%%%%%%%%%%%%%%%%%%%%%%%%%%%%%%%%%%%%%%%%%%%%%%%%%%
%%
%W  intro.tex             GAP documentation                  Bettina Eick
%%
%H  $Id: intro.tex,v 1.9 2003/10/24 12:05:36 werner Exp $
%%

%%%%%%%%%%%%%%%%%%%%%%%%%%%%%%%%%%%%%%%%%%%%%%%%%%%%%%%%%%%%%%%%%%%%%%%%%%%
\Chapter{Introduction}

A Lie $p$-ring is a nilpotent Lie ring of $p$-power order for a prime $p$. 
This package contains some algorithms for Lie $p$-rings and it gives access 
to the database of Lie $p$-rings of order at most $p^7$ as determined by 
Mike Newman, Eamonn O'Brien and Michael Vaughan-Lee, see \cite{NOV04} and 
\cite{OVL05}.
\medskip

For this purpose, this package introduces a datastructure for Lie $p$-rings. 
For a fixed prime $p$, this datastructure is quite similar to power-conjugate 
presentations for finite $p$-groups. The datastructure introduced here also 
allows to define Lie $p$-rings in which the presentation contains 
indeterminates. In particular, the prime $p$ is allowed to be an 
indeterminate. Such Lie $p$-rings are called {\it generic}; they are 
described in more detail in Chapter 2 of this manual.
\medskip

The package then defines for each $n \in \{1, \ldots, 7\}$ a (finite) list 
of generic presentations of Lie $p$-rings. For each prime $p \geq 5$, each 
of the generic Lie $p$-rings gives rise to a family of Lie $p$-rings over 
the considered prime $p$ by specialising the indeterminates to a certain list 
of values. The resulting lists of Lie $p$-rings provides a complete and 
irredundant set of isomorphism type representatives of the Lie $p$-rings of 
order $p^n$. The generic Lie $p$-rings of $p$-class at most 2 can also be 
considered for the prime $p=3$ and yield a list of isomorphism type 
representatives for the Lie $p$-rings of order $3^n$ and class at most $2$.
(The $p$-class of a Lie ring is the length of the lower exponent-$p$ central
series.)
\medskip

The Lazard correspondence induces a one-to-one correspondence between the
Lie $p$-rings of order $p^n$ and class less than $p$ and the $p$-groups of 
order $p^n$ and class less than $p$. This package provides a function to 
evaluate this correspondence; this function has been implemented by Willem
de Graaf. It uses the Liering package \cite{CdG10}. 
\medskip

The Lazard correspondence has been used to check the correctness of the
database of Lie $p$-rings: for various small primes it has been checked
that the Lie $p$-rings of this database define non-isomorphic finite 
$p$-groups.


