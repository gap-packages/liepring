%%%%%%%%%%%%%%%%%%%%%%%%%%%%%%%%%%%%%%%%%%%%%%%%%%%%%%%%%%%%%%%%%%%%%%%%%
%%
%W  advan.tex             GAP documentation                  Bettina Eick
%%
%%

%%%%%%%%%%%%%%%%%%%%%%%%%%%%%%%%%%%%%%%%%%%%%%%%%%%%%%%%%%%%%%%%%%%%%%%%%%%
\Chapter{Advanced functions for Lie p-rings}

This chapter described a few more advanced functions available for 
generic Lie p-rings. 

%%%%%%%%%%%%%%%%%%%%%%%%%%%%%%%%%%%%%%%%%%%%%%%%%%%%%%%%%%%%%%%%%%%%%%%%%%%
\Section{Schur multipliers}

The package contains a method to determine the Schur multiplier of the
Lie p-rings in the family defined by a generic Lie p-ring. 

\> LiePSchurMult( L )

The function
takes as input a generic Lie p-ring and determines a list of possible
Schur multipliers, each described by its abelian invariants, and for each
entry in the list there is a description of those parameters which 
give the considered entry. This description consists of two lists 
'units' and 'zeros' of rational functions over the parameters of the 
Lie p-ring. The parameters described by these lists are which evaluate
to zero for each rational function in 'zeros' and evaluate not to zero
for each rational function in 'units'.

\beginexample
gap> L := Filtered(LL, x -> Length(ParametersOfLiePRing(x))=7)[1]; 
<LiePRing of dimension 7 over prime p with parameters 
[ w, x, y, z, t, s, u, v ]> 
gap> ss := LiePSchurMult(L);
[rec( norm := [ p, p, p, p ], 
   units := [ x^2-w, y^2-w, z^2-w, t^2-w, s^2-w, u^2-w, v^2-w, s, w*x-s*v ], 
   zeros := [  ] ), 
 rec( norm := [ p, p, p, p ], 
   units := [ x^2-w, y^2-w, z^2-w, t^2-w, s^2-w, u^2-w, v^2-w, s, -z*s+y ],
   zeros := [ w*x-s*v ] ), 
 rec( norm := [ p, p, p, p ], 
   units := [ x^2-w, y^2-w, z^2-w, t^2-w, s^2-w, u^2-w, v^2-w, s, -1/2*u ],
   zeros := [ -z*s+y, w*x-s*v ] ), 
 rec( norm := [ p, p, p, p ], 
   units := [ x^2-w, y^2-w, z^2-w, t^2-w, s^2-w, u^2-w, v^2-w, s, -1/2*s, 
              x*s+2*t-v ], 
   zeros := [ u, -z*s+y, w*x-s*v ] ), 
 rec( norm := [ p, p, p, p ], 
   units := [ x^2-w, y^2-w, z^2-w, t^2-w, s^2-w, u^2-w, v^2-w, s, 
             -1/2*z-1/2*v ], 
   zeros := [ u, -z*s+y, x*s+2*t-v, 1/2*s^2*v+w*t-1/2*w*v, w*x-s*v ] ), 
 rec( norm := [ p, p, p, p, p ], 
   units := [ x^2-w, y^2-w, z^2-w, t^2-w, s^2-w, u^2-w, v^2-w, s ], 
   zeros := [ v, u, t, z, y, x ] ), 
 rec( norm := [ p, p, p, p^2 ], 
   units := [ x^2-w, y^2-w, z^2-w, t^2-w, s^2-w, u^2-w, v^2-w, s, -v, 2*t ], 
   zeros := [ u, z+v, s*v+y, x*s+2*t-v, 1/2*s^2*v+w*t-1/2*w*v, w*x-s*v ] ) ]
\endexample

Note that $w$ is a primitive root and hence a polynomial $a^2-w$ for each 
parameter $a$ is non-zero for every value of $a$. Thus, as an example, the
above example shows that the Schur multiplier $[p,p,p,p,p]$ is obtained if 
$s \neq 0$ and $v=u=t=z=y=x =0$.

The package also contains a function that tries to determine the numbers of 
values of the parameters satisfying the conditions of a description of
a Schur multiplier. This succeeds in many cases and returns a polynomial
in $p$ in this case. If it does not succeed then it returns fail.

\> ElementNumber( pp, units, zeros )

We continue the above example.

\beginexample
gap> pp := ParametersOfLiePRing(L);
[ x, y, z, t, s, u, v ]
gap> numb := List(ss, x -> ElementNumber(pp, x.units, x.zeros));
[ p^7-2*p^6+p^5, 
  p^6-2*p^5+p^4, 
  p^5-2*p^4+p^3, 
  p^4-2*p^3+p^2, 
  p^3-2*p^2+p, 
  p-1, 
  p^2-2*p+1 ]
gap> Sum(numb);
p^7-p^6
gap> invs := RingInvariants(L);
rec( units := [ x^2-w, y^2-w, z^2-w, t^2-w, s^2-w, u^2-w, v^2-w, s ], 
     zeros := [  ] )
gap> ElementNumber(pp, invs.units, invs.zeros);
p^7-p^6
\endexample

There also is a function which automatically sums up the numbers 
associated to a certain Schur multiplier. 

\> ElementNumbers( pp, ss )

In the above example this applies as follows.

\beginexample
gap> ElementNumbers(pp, ss);
rec( norms := [ [ p, p, p, p ], [ p, p, p, p, p ], [ p, p, p, p^2 ] ], 
     numbs := [ p^7-p^6-p^2+p, p-1, p^2-2*p+1 ] )
\endexample

%%%%%%%%%%%%%%%%%%%%%%%%%%%%%%%%%%%%%%%%%%%%%%%%%%%%%%%%%%%%%%%%%%%%%%%%%%%
\Section{Automorphism groups}

The package contains a function that determines a description for the
automorphism group of a generic Lie p-ring.

\> FindAutos( L )

We start a new example using the Lie p-ring 7.38 of the library.

\beginexample
gap> ViewPCPresentation(L);
p*l1 = l4
p*l2 = l5
p*l3 = l6
p*l4 = x*l7
p*l5 = y*l6 + l7
[l2,l1] = l3
[l3,l1] = l6
[l3,l2] = l7
[l4,l2] = -l6
[l5,l1] = l6
gap> FindAutos(L);
rec( 
  auto := [ [ 1, 0, A13, A14, A15, A16, A17 ], 
            [ 0, 1, A23, A24, A25, A26, A27 ] ], 
  eqns := [  ], 
  vars := [ A11, A12, A13, A14, A15, A16, A17, A21, A22, A23, 
            A24, A25, A26, A27, D ] )
\endexample

This description asserts that the Lie p-ring $L$ is generated by the
first two basis elements $l1, l2$. The entry 'auto' contains the 
possible images of these two generators, where $Aij$ are indeterminates
that can take any value in $0, \ldots, p-1$. The entry 'eqns' contains
additional restrictions on these indeterminates. The entry 'vars' lists
the indeterminates occuring in the description of the automorphism group.

